%!TEX TS-program = xelatex
%!TEX encoding = UTF-8 Unicode
% Awesome CV LaTeX Template for Cover Letter
%
% This template has been downloaded from:
% https://github.com/posquit0/Awesome-CV
%
% Authors:
% Claud D. Park <posquit0.bj@gmail.com>
% Lars Richter <mail@ayeks.de>
%
% Template license:
% CC BY-SA 4.0 (https://creativecommons.org/licenses/by-sa/4.0/)
%


%-------------------------------------------------------------------------------
% CONFIGURATIONS
%-------------------------------------------------------------------------------
% A4 paper size by default, use 'letterpaper' for US letter
\documentclass[11pt, a4paper]{awesome-cv}

% Configure page margins with geometry
\geometry{left=1.4cm, top=.8cm, right=1.4cm, bottom=1.8cm, footskip=.5cm}

% Specify the location of the included fonts
\fontdir[fonts/]

% Color for highlights
% Awesome Colors: awesome-emerald, awesome-skyblue, awesome-red, awesome-pink, awesome-orange
%                 awesome-nephritis, awesome-concrete, awesome-darknight
\colorlet{awesome}{awesome-red}
% Uncomment if you would like to specify your own color
% \definecolor{awesome}{HTML}{CA63A8}

% Colors for text
% Uncomment if you would like to specify your own color
% \definecolor{darktext}{HTML}{414141}
% \definecolor{text}{HTML}{333333}
% \definecolor{graytext}{HTML}{5D5D5D}
% \definecolor{lighttext}{HTML}{999999}

% Set false if you don't want to highlight section with awesome color
\setbool{acvSectionColorHighlight}{true}

% If you would like to change the social information separator from a pipe (|) to something else
\renewcommand{\acvHeaderSocialSep}{\quad\textbar\quad}


%-------------------------------------------------------------------------------
%	PERSONAL INFORMATION
%	Comment any of the lines below if they are not required
%-------------------------------------------------------------------------------
% Available options: circle|rectangle,edge/noedge,left/right
\photo[circle,noedge,left]{./examples/profile}
\name{Ray}{Matsumoto}
\position{Scientific Researcher{\enskip\cdotp\enskip}Data Scientist}
\address{1700 State Street Apt 412, Nashville, TN, 37203, USA}

\mobile{(+1) (843) 475-8628}
\email{ray.a.matsumoto@vanderbilt.edu}
\homepage{www.raymatsumoto.com}
\github{rmatsum836}
% \gitlab{gitlab-id}
% \stackoverflow{SO-id}{SO-name}
% \twitter{@twit}
% \skype{skype-id}
% \reddit{reddit-id}
% \medium{madium-id}
% \googlescholar{googlescholar-id}{name-to-display}
%% \firstname and \lastname will be used
% \googlescholar{googlescholar-id}{}
% \extrainfo{extra informations}


%-------------------------------------------------------------------------------
%	LETTER INFORMATION
%	All of the below lines must be filled out
%-------------------------------------------------------------------------------
% The company being applied to
\recipient
  {Research and Innovation Group}
  {Solvay \\Alpharetta,
  GA, USA}
% The date on the letter, default is the date of compilation
\letterdate{\today}
% The title of the letter
\lettertitle{Job Application for R\&I Data Scientist}
% How the letter is opened
\letteropening{To whom it may concern,}
% How the letter is closed
\letterclosing{Sincerely,}
% Any enclosures with the letter
%\letterenclosure[Attached]{Curriculum Vitae}


%-------------------------------------------------------------------------------
\begin{document}

% Print the header with above personal informations
% Give optional argument to change alignment(C: center, L: left, R: right)
\makecvheader[R]

% Print the footer with 3 arguments(<left>, <center>, <right>)
% Leave any of these blank if they are not needed
\makecvfooter
  {\today}
  {Ray Matsumoto~~~·~~~Cover Letter}
  {}

% Print the title with above letter informations
\makelettertitle

%-------------------------------------------------------------------------------
%	LETTER CONTENT
%-------------------------------------------------------------------------------
\begin{cvletter}

I am an enthusiastic researcher writing to express my interest in
    joining the R\&I group at Solvay.
    I am particularly excited about the opportunity to use data science
    algorithms and AI to support scientific computing acceleration.

    I believe my background would make me a great fit within the group.
    I received my undergraduate degree in Materials Science and Engineering, and
    have taken coursework on polymers, mechanics of materials, and organic
    chemistry.

    During my graduate research in molecular modeling and computational
    chemistry, I developed a strong foundation in
    Python and data analysis, regularly developing software to implement
    statistical methods for understanding simulation data.  
    For instance, I developed an automated workflow on a computer cluster to run simulations for over 
    400 chemical systems, conduct various analysis techniques, and compile the data into CSV format.  
    Further, I have demonstrated success in communicating key insights and
    findings through ten peer-reviewed journal articles and two conference
    presentations.

    In regards to my technical expertise in data science, I have completed
     several projects leveraging the Python data science ecosystem.  I developed a
     random forest regression model to predict properties of electrolytes using
     the cheminformatics library, RDKit.  From these projects, I also have
     experience validating data quality, cleaning data, and measuring the
     accuracy of models.

    My biggest strength, however, is in effective collaboration within diverse
    teams.  As an example, I led a project with three graduate students and one
    postdoctoral researcher in which we demonstrated how reproducible simulation workflows could be
    developed with open-source software.  My participation in collaborative
    research resulted in ten peer-reviewed journal articles and two conference presentations.
    My understanding is that the above skills would be
    transferable with the tasks performed as a R\&I Data Scientist.

    Data-driven approaches have shown immense promise for enabling discoveries
    in scientific research.
    It would be a privilege to support the R\&I efforts as a
    Data Scientist at Solvay.

    I appreciate your time and consideration.

\end{cvletter}


%-------------------------------------------------------------------------------
% Print the signature and enclosures with above letter informations
\makeletterclosing

\end{document}
